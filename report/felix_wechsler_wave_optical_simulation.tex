\documentclass[a4paper,12pt]{article}
\usepackage{geometry}

% Set small borders with the geometry package
\geometry{
  top=3cm,     % Top margin
  bottom=3cm,  % Bottom margin
  left=2.5cm,    % Left margin
  right=2.5cm,   % Right margin
}

\usepackage{amsmath}  % For mathematical symbols and equations
\usepackage{amsfonts} % For extra fonts
\usepackage{graphicx} % For including graphics
\usepackage{hyperref} % For hyperlinks (optional)


\usepackage[
    backend=biber,
    sorting=none,
]{biblatex}
\addbibresource{references.bib}

\usepackage{setspace} % For line spacing
% \onehalfspacing


\title{Wave Optical Simulation of Thick Optical Elements}
\author{Felix Wechsler}
\date{\today}
\setlength\parindent{0pt}
\begin{document}
\maketitle
\thispagestyle{empty}

\begin{abstract}
In this proposal, we outline our approach to simulating optical lenses using a wave optics framework. We aim to apply and compare various methods for accurately modeling wave propagation through thick lenses, evaluating each method's performance.
\end{abstract}


\section{Introduction}
Ray optical tools are standard tools to characterize optical elements. However, going to smaller scales the wave nature of light is not negligible.
In this work, we study if simple wave optical algorithms such as the angular spectrum method of plane waves (AS) are suitable and practical to
simulate thick optical elements.
In a concrete example of a spherical lens, we demonstrate how the multi-slice angular spectrum fails to predict geometrical parameters such as the focal length.
We explore how more advanced methods such as the modified wave propagation method \cite{schmidt2016wave} are able to predict those properties for an 
acceptable computational complexity. 


\section{Theoretical basis}
This section introduces briefly the main concepts behind the different approaches.
For more details, please see the listed references.
\subsection{Angular spectrum method of plane waves}
The angular spectrum method of plane waves (AS) is a solution to the homogenous, isotropic Helmholtz equation which describes free space wave propagation
in a medium with constant and isotropic refractive index $n$.
The mathematical form is
\begin{align}
\psi_{z}&=\mathcal{F}^{-1}\left[\mathcal{F}\left[\psi_{0}\right] \cdot H_{\textrm{AS}}\right],\\
H_{\textrm{AS}}\left(f_{x},f_{y}\right)&=\exp\left(i2\pi z\sqrt{\frac{n^2}{\lambda^{2}}-f_{x}^{2}-f_{y}^{2}}\right)
\end{align}
where $\psi_0$ is the incoming scalar electrical field, $\mathcal{F}$ the 2D Fourier transform, $\lambda$ the wavelength in vacuum, $z$ the propagation distance and $f_{x/y}$ the spatial frequencies in Fourier domain.
Computationally, this can be implemented with two Fast Fourier transform (FFT) algorithms albeit circular wrap-around artifacts of the FFT should be avoided
with the band-limited version of the AS \cite{matsushima2009band}.


\subsection{Multi-slice}
Multi-slice propagation (MS) is an approach to solve wave-propagation in an inhomogeneous (spatially varying refractive index) medium. 
The angular-spectrum method can be extended to multi-slice propagation by dividing the medium into small slices along the optical axis and applying the angular spectrum method to each slice \cite{https://doi.org/10.1107/S0365110X57002194,Li_Wojcik_Jacobsen_2017}. Between each slice, the effect of the medium is 
described as a \textit{thin medium} which allows to include the effect of the medium as a phase shift to the electrical field.
For example, propagation through a medium $n(x,y,z)$ can be described as a series of $N_z$ AS steps $\mathcal{A}$ which corresponding phase shift in between.
The total propagation distance $z$ is split into $\Delta z = z / N_z$ steps.
$n_0$ is an average refractive index of the medium. and $k_0$ is the wavenumber in this medium $n_0$.
\begin{align}
    \psi(x,y,z) = \underbrace{\mathcal{A}\bigg[\exp\left(i k_0 n(x,y, N_z \cdot \Delta z)\mathcal{A}\big[\exp\left(i k_0 n(x,y, (N_z - 1) \cdot \Delta z) \right) \cdots \psi(x,y, 0) \big] \right) \bigg]}_{N_z \text{ times application of }\mathcal{A}} 
\end{align}
In \label{sec:sim} we will show that this method fails to predict geometrical properties of thick optical elements.
The reason is, that if the medium $n$ deviates strongly from $n$, the MS applies only phase shifts to the electrical field.
Let us do the following Gedankenexperiment that $n_0 = 1$ but $n=1.5$. If we would
propagate a Gaussian beam through this medium, MS would only do constant phase shifts since $n$ is constant at all $(x,y,z)$. But, a constant phase shift will not change any curvature of the beam. But, the curvature change of a Gaussian beam depends on the refractive index the beam is propagating through. It can be seen as if the MS is doing a correct phase change but it's not changing the wavelength accordingly. 
In \autoref{sec:sim} we will see how this leads to significant errors for a thick lens.

\subsubsection{Modified wave propagation method}
To overcome the shortcomings of the MS, the modified wave propagation method (MWPM) was proposed \cite{schmidt2016wave}.

In integral form, the wave propagation can be written as
\begin{align}
\label{wpm}
    \psi(x,y,z+\Delta z) &= \frac{1}{2\pi} \int \widetilde{\psi}(k_x, k_y, z) \, e^{ik_z(k_x, k_y, x, y) \Delta z} \, e^{i (k_x x + k_y y)} \, \mathrm{d}k_x \, \mathrm{d}k_y\\
    k_z(k_x, k_y, x, y) &= \sqrt{k_0^2 n^2 \left( x,y,z + \frac{\Delta z}{2} \right) - k_x^2 - k_y^2}
\end{align}
where $\widetilde{\psi}$ is the Fourier transform of the electrical field $\psi$.
The spatially varying $k_z(k_x, k_y, x, y)$ forbids to express \autoref{wpm} as a Fourier transform integral.
However, the MWPM argues to express this integral as a sum of piecewise constant functions which are stitched together at regions where $n$ changes.
This assumes there is a finite number of regions where $n$ suddenly changes such as in the case of air and lens interfaces.
\begin{align}
    I_m^z(x,y) &=
    \begin{cases}
    1 & n_z(x,y) = n_m, \\
    0 & n_z(x,y) \ne n_m,
    \end{cases}, \\
    \psi(x,y,z+\Delta z) &= \sum_m I_m^z(x,y) \mathcal{F}^{-1} \left\{ e^{i k_z^m(k_x,k_y)\Delta z} \mathcal{F} \left\{\psi(x,y,z) \right\} \right\},\\
    k_z^m(k_x, k_y) &= \sqrt{k_0^2 n_m^2 - k_x^2 - k_y^2} + \kappa(k_x, k_y).
\end{align}
$k_z^m$ is the wavenumber in the medium $m$.
The MWPM can be implemented with one MS step for each medium $m$.


\subsubsection{Hankel transform based methods}
As the last concept we introduce a Hankel transform based wave propagation \cite{Guizar-Sicairos_Gutiérrez-Vega_2004}.
The Hankel transform is a generalization of the Fourier transform to cylindrical coordinates. It can be used to describe the propagation of a wave through a medium with cylindrical symmetry such as a spherical lens.

Let us recall that the Fourier transform can be written as

\begin{equation}
\mathcal{F}[f](k_x, k_y) = \int_{-\infty}^{\infty} F(x,y) \exp(i (k_x \cdot x + k_y \cdot y)) \,\mathrm{d}x\, \mathrm{d}y = \int_{-\infty}^{\infty} F(x,y) \exp(i \vec k \cdot \vec r) \,\mathrm{d}x\, \mathrm{d}y
\end{equation}
If we transform this do polar coordinates $(r, \theta)$ and $(\kappa, \phi)$ we obtain
\begin{equation}
\mathcal{F}[f](\kappa, \phi) = \int_{0}^{\infty} \int_{0}^{2\pi} r f(r, \theta)\cdot \exp(i \cdot \cos(\theta - \phi) \kappa r)  \,\mathrm{d}\theta\, \mathrm{d}r
\end{equation}
where $\kappa = \sqrt{k_x^2 + k_y^2}$ and $r=\sqrt{x^2 + y^2}$

We can now use
\begin{equation}
\exp(i x \cdot \sin(\theta)) = \sum_{n=-\infty}^{\infty} J_n(x) \cdot \exp(i n \theta)
\end{equation}
where $J_n$ is the nth-order Bessel function of the first kind.

After integration (if $f$ has no $\theta$ dependency) over $\theta$ it only remains

\begin{equation}
\mathcal{F}[f](\kappa, \phi) = 2\pi \int_{0}^{\infty} r \cdot f(r) \cdot J_0(\kappa \cdot r) \, \mathrm{d}r = \mathcal{H}[f](\rho)
\end{equation}
where $\mathcal{H}$ is the Hankel transform.
With the Fast Hankel transform \cite{Guizar-Sicairos_Gutiérrez-Vega_2004} we can now efficiently calculate the 2D field propagation with a 1D matrix-vector-product.If required, the radial 1D field can be reconverted to cartesian coordinates. 

\section{Simulation Results}
In this part, we present our main results of the simulation study. 
\label{sec:sim}

\section{Conclusion}


\printbibliography

\end{document}


